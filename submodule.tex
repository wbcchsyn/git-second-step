% Copyright 2014 Yoshida Shin
% 
% This is part of "Git Second Step."
% 
% This program is free software: you can redistribute it and/or modify
%     it under the terms of the GNU General Public License as published by
%     the Free Software Foundation, either version 3 of the License, or
%     (at your option) any later version.
% 
%     This program is distributed in the hope that it will be useful,
%     but WITHOUT ANY WARRANTY; without even the implied warranty of
%     MERCHANTABILITY or FITNESS FOR A PARTICULAR PURPOSE.  See the
%     GNU General Public License for more details.
% 
%     You should have received a copy of the GNU General Public License
%     along with this program.  If not, see <http://www.gnu.org/licenses/>.

\begin{frame}{}{}
  git の入れ子管理
\end{frame}

\begin{frame}[t]{git の入れ子管理}{子 repository の使用}

  \begin{columns}

    \begin{narrowcolumn}
      {\small
        \onslide*<1-5>{repo1}\onslide*<6-20>{\color{blue}repo1\color{black}}\onslide*<21->{\color{green}repo1\color{black}}

        ~~{\vbar}-- libs

        \onslide*<3->{
          ~~{\vbar}~~~~{\vbar}-- \alert<9->{repo2}

          ~~{\vbar}~~~~{\vbar}~~~~{\vbar}-- ...
        }

        ~~{\vbar}~~~~{\vbar}-- ...

        \onslide*<3->{~~{\vbar}-- .gitmodules}

        ~~{\vbar}-- ...
        \vspace{3ex}

        \onslide*<11->{
          \onslide*<-23>{\color{blue}clone\color{black}}\onslide*<24->{\color{green}clone\color{black}}

          ~~{\vbar}-- libs

          ~~{\vbar}~~~~{\vbar}-- \alert<26->{repo2}

          \onslide*<15->{~~{\vbar}~~~~{\vbar}~~~~{\vbar}-- ...}

          ~~{\vbar}~~~~{\vbar}-- ...

          ~~{\vbar}-- .gitmodules

          ~~{\vbar}-- ...
        }
      }
    \end{narrowcolumn}

    \begin{widecolumn}

      \onslide*<1>{
        別 repository の repo2 を

        repo1 の中で入れ子管理する

        (repo1 の submodule にする)
      }

      \onslide*<2-3>{
        \alert{workspace の top ディレクトリ}で
        \vspace{2ex}

        \$ git submodule add \textit{URI} libs/repo2
        \vspace{4ex}

        \onslide<3->{
          repo2 が clone され、

          .gitmodules が作成される
        }
      }

      \onslide*<4-6>{
        .gitmodules を commit すれば

        submodule 化完了
        \vspace{2ex}

        \onslide*<5-6>{
          \$ git add .gitmodules

          \$ git commit
        }
      }

      \onslide*<7-9>{
        repo2 以下のディレクトリで

        何をやっても

        repo1 には影響を与えない

        (repo2 以下のディレクトリでは

        ~repo1 は見えない)
        \vspace{2ex}

        \onslide*<8->{
          \$ cd libs/repo2

          \$ touch a

          \$ git add a

          \$ git commit

          \$ git push origin master
        }
        \vspace{2ex}
      }

      \onslide*<10-12>{
        repo1 を clone してみる
        \vspace{2ex}

        \$ git clone repo1 clone
        \vspace{4ex}

        \onslide*<12>{repo2 の中身は空}
      }

      \onslide*<13-16>{
        repo2 を clone するには、別途 submodule update が必要
        \vspace{2ex}

        \onslide*<14->{
          \$ git submodule update {\bslash}

          ~~~~{\dhyphen}init {\dhyphen}recursive

          (全 submodule を clone)
          \vspace{1ex}

          \onslide*<16->{
            {\dhyphen}init

            初めて update する

            submodule がある時に必要
            \vspace{2ex}

            {\dhyphen}recursive

            repo2 がさらに別の submodule を

            管理している場合に必要
          }
        }
      }

      \onslide*<17-18>{
        でも、repo2 の HEAD が古い commit を指している
        \vspace{2ex}

        \onslide*<18>{
          これは、repo1 が submodule の

          バージョンまで管理しているから

          (repo1 で commit した時点の

          ~repo2 の commit へ

          ~checkout している)
        }
        \vspace{2ex}
      }
      \onslide*<19-21>{
        repo2 のバージョンを上げるには、

        repo1 で repo2 の新しいバージョンを commit
        \vspace{2ex}

        \onslide*<20->{
          \$ git add libs/repo2

          \$ git commit

          \$ git push origin master
        }
      }

      \onslide*<22->{
        clone を更新して

        再度 submodule update を行えば

        clone でも repo2 の HEAD が

        切り替わる
        \vspace{2ex}

        \onslide*<23->{
          \$ git fetch {\dhyphen}all -p

          \$ git merge remotes/origin/master

          \onslide*<25->{
            \$ git submodule update {\dhyphen}recursive
            \vspace{2ex}

            (初めて submodule update する

            ~repository は無いので

            ~{\dhyphen}init は不要)
          }
        }
      }
    \end{widecolumn}

  \end{columns}
\end{frame}

