% Copyright 2014 Yoshida Shin
% 
% This is part of "Git Second Step."
% 
% This program is free software: you can redistribute it and/or modify
%     it under the terms of the GNU General Public License as published by
%     the Free Software Foundation, either version 3 of the License, or
%     (at your option) any later version.
% 
%     This program is distributed in the hope that it will be useful,
%     but WITHOUT ANY WARRANTY; without even the implied warranty of
%     MERCHANTABILITY or FITNESS FOR A PARTICULAR PURPOSE.  See the
%     GNU General Public License for more details.
% 
%     You should have received a copy of the GNU General Public License
%     along with this program.  If not, see <http://www.gnu.org/licenses/>.

\begin{frame}{}{}
  過去の履歴を書き換える
\end{frame}


\begin{frame}[t]{過去の履歴を書き換える}{}
  SCM の悪い履歴とは?

  \onslide*<2->{
    \begin{itemize}
    \item 紆余曲折がそのままでノイズ多
      \begin{enumerate}
      \item method A, B, C を実装
      \item method B のバグ修正
      \item 結局、method A, B は削除
      \end{enumerate}
      \onslide*<3->{最初から method C だけ書いてよ}
      \vspace{2ex}

      \onslide*<4->{
      \item 動作確認出来ない的 commit

        (括弧の閉じ忘れで動かない等)
      }

    \end{itemize}
  }
\end{frame}


\begin{frame}[t]{過去の履歴を書き換える}{}
  悪い履歴を残さないためには?
  \vspace{4ex}

  \onslide*<2->{
    commit の度にしっかりテスト、吟味する
    \onslide*<3->{
      \begin{enumerate}
      \item 副作用として、commit のハードルが上がる
        \onslide*<4->{\item 大きい commit を少数行うようになる}
        \onslide*<5->{
        \item 突き詰めると、commit 1個 だけ(完成品のみ)

          に行き着く

          super water fall 万歳
        }
      \end{enumerate}
    }
  }
  \vspace{4ex}

  \onslide*<6->{少し大げさかもしれないが、何かおかしくない?}
\end{frame}


\begin{frame}[t]{過去の履歴を書き換える}{}
  悪い履歴を残さない、もう一つの方法は

  過去の履歴の修正
  \vspace{4ex}

  \onslide*<2->{
    \begin{itemize}
    \item 古い commit のバグが、後になって発覚

      \onslide*<3->{バグ修正 commit の追加より、古い commit の修正}
      \vspace{2ex}

      \onslide*<4->{
      \item 最終的に、特定の commit が不要になった

        \onslide*<5->{
          不要部分の削除 commit より、

          該当箇所を実装した commit 自体の削除
        }
      }
    \end{itemize}
  }
  \vspace{4ex}

  \onslide*<6->{
    git なら、履歴の修正が可能

    最終的に悪い履歴が残らなければ良い
  }
\end{frame}


\begin{frame}[t]{過去の履歴を書き換える}{}
  ただし、履歴を修正してはいけない branch も有る
  \vspace{2ex}

  修正して良い例
  \begin{itemize}
  \item local repository の branch
  \item クローズドな repository 内で

    閲覧、更新者が限られた branch
    \begin{itemize}
    \item レビューのため、個人名をつけた branch
    \item 担当者のアサインされた topic branch
    \end{itemize}
  \end{itemize}
  \vspace{2ex}

  修正してはいけない例
  \begin{itemize}
  \item 不特定多数の人が閲覧可能な branch
  \item その他、一般的な remote reposiotry の branch
  \end{itemize}
\end{frame}


\begin{frame}[t]{過去の履歴を書き換える}{}
  SCM の履歴はドキュメンタリーでは無い

  不要な物は捨てて、必要十分な情報のみ残す
  \vspace{2ex}

  しかし、失敗を恐れて commit を控えれば、

  SCM の機能を十分に使わない事になる
  \vspace{4ex}

  要するに、最終的に複数人で共有する branch へ

  push する時までに直せば良い

  それまでは気軽に commit するべき

\end{frame}


\begin{frame}[t]{過去の履歴を書き換える}{}

  \begin{columns}

    \begin{narrowcolumn}

      \begin{center}
        \begin{tikzpicture}[x=1em, y=1ex]

          \node (G) [commit] at (0, 24) {G};
          \node (F) [commit] at (0, 20) {F};
          \node (E) [commit] at (0, 16) {E};
          \node (D) [commit] at (0, 12) {D};
          \node (C) [commit] at (0, 8) {C};
          \node (B) [commit] at (0, 4) {B};
          \node (A) [commit] at (0, 0) {A};

          \draw (F) -- (G);
          \draw (E) -- (F);
          \draw (D) -- (E);
          \draw (C) -- (D);
          \draw (B) -- (C);
          \draw (A) -- (B);

        \end{tikzpicture}

        {\footnotesize{(G \alert{$\leftarrow$ HEAD})}}
      \end{center}

    \end{narrowcolumn}

    \begin{widecolumn}

      直近 6個の履歴 (B, C, D, E, F, G) を見直す場合
      \vspace{2ex}

      \$ git rebase -i \textit{A}
      \vspace{2ex}

      を実行
      \vspace{2ex}

      (B の1個前の commit を指定する)

      (commit の指定方法は branch, tag, ハッシュ値等)
    \end{widecolumn}

  \end{columns}

\end{frame}


\begin{frame}[t]{過去の履歴を書き換える}{}

  \begin{columns}

    \begin{halfcolumn}
      \code{\onslide*<-4>{\alert<2>{pick}}\onslide*<5->{\alert{r}}~824c7c9~message~B

        \alert<2>{pick}~36dc484~message~C

        \onslide*<-6>{\alert<2>{pick}}\onslide*<7->{\alert{f}}~9a7c30a~message~D

        \onslide*<9->{\alert{\#}}\alert<2>{pick}~0ed3c2e~message~E

        \onslide*<-10>{
          \alert<2>{pick}~b7a1a35~message~F

          \alert<2>{pick}~ac859a3~message~G
        }

        \onslide*<11->{
          \alert{f}~ac859a3~message~G

          pick~b7a1a35~message~F
        }

        ...
      }

    \end{halfcolumn}

    \begin{halfcolumn}

      \onslide*<1>{
        エディタが開き、

        左のような画面が表示
        \vspace{2ex}

        上が古く、下が新しい

        (git log と逆なので注意)
      }

      \onslide*<2>{
        この画面では、

        どの履歴を編集するか指定
        \vspace{4ex}

        実際に履歴を編集するのは後
        \vspace{2ex}

        書き換えるのは主に\alert{この列}
      }

      \onslide*<3>{
        ``pick'' は「何も手を加えない」という意味
      }

      \onslide*<4-5>{
        B の commit message を

        変更するには
        \vspace{4ex}

        B の ``pick'' を ``r'' に変更

        (または、``reword'' に変更)
      }

      \onslide*<6-7>{
        D の commit を

        前の commit (C) と

        1個にまとめるには
        \vspace{4ex}

        D の ``pick'' を ``f''に変更

        (または、``fixup'' に変更)
      }

      \onslide*<8-9>{
        E の commit を削除するには

        E の行をコメントアウト
        \vspace{2ex}

        (履歴を残さない事以外は

        git revert と似ている)
      }

      \onslide*<10-11>{
        G の commit を
        前の commit (C+D) と

        1個にまとめるには
        \vspace{4ex}

        F と G の行を入れ替えて

        G の ``pick'' を ``f'' に変更する
      }

      \onslide*<12>{
        この状態でエディタを

        保存、終了
      }

    \end{halfcolumn}

  \end{columns}

\end{frame}

\begin{frame}[t]{過去の履歴を書き換える}{}
  \begin{columns}

    \begin{halfcolumn}

      \begin{center}
        \begin{tikzpicture}[x=1em, y=1ex]

          \node (G) [commit] at (0, 24) {G};
          \node (F) [commit] at (0, 20) {F};
          \node (E) [commit] at (0, 16) {E};
          \node (D) [commit] at (0, 12) {D};
          \node (C) [commit] at (0, 8) {C};
          \node (B) [commit] at (0, 4) {B};
          \node (A) [commit] at (0, 0) {A};

          \onslide*<-7>{
            \draw (E) -- (F);
          }

          \draw (D) -- (E);

          \onslide*<-5>{
            \draw (F) -- (G);
            \draw (C) -- (D);
            \draw (B) -- (C);
          }
          \onslide*<-3>{\draw (A) -- (B);}


          \onslide<8->{\node (F') [commit] at (6, 20) {F'};}
          \onslide<6->{\node (C') [commit] at (6, 16) {C+D+G};}
          \onslide<4->{\node (B') [commit] at (6, 4) {B'};}

          \onslide*<8->{
            \draw [line width=3pt, color=blue] (E) -- (F);
            \draw [line width=3pt, color=blue] (C') -- (F');
          }

          \onslide*<6->{
            \draw [line width=3pt, color=green] (B') -- (C');
            \draw [line width=3pt, color=green] (F) -- (G);
            \draw [line width=3pt, color=green] (C) -- (D);
            \draw [line width=3pt, color=green] (B) -- (C);
          }

          \onslide*<4->{
            \draw [line width=3pt, color=red] (A) -- (B');
            \draw [line width=3pt, color=red] (A) -- (B);
          }
        \end{tikzpicture}
      \end{center}

    \end{halfcolumn}

    \begin{halfcolumn}

      \onslide*<1>{
        まず、HEAD を A に切り替える
      }

      \onslide*<2>{
        エディタが開き、

        B の commit message

        再入力
      }

      \onslide*<3-4>{
        B のパッチを

        A に適用
      }

      \onslide*<5-6>{
        C+D+G のパッチを

        B' へ適用
        \vspace{4ex}

        Commit message は

        C の物を使用
      }

      \onslide*<7-8>{
        F は pick なので、

        何もせず F のパッチを

        C+D+G に適用
      }


      \onslide*<9-10>{
        その他、

        r と f を一度に実行する

        s (または squash)
        \vspace{2ex}

        \onslide*<10>{
          rebase を一回止める

          e (または edit)

          (この間に色々出来る)
          \vspace{2ex}

          もある
        }
      }

      \onslide*<11->{
        でも、個人的には

        s や e が役に立った

        記憶が少ないので

        説明省略
      }
      \vspace{2ex}

      \onslide*<12->{気になる人は、man で調べて}

    \end{halfcolumn}

  \end{columns}

\end{frame}


\begin{frame}[t]{過去の履歴を書き換える}{}

  ところで、
  \vspace{2ex}

  \onslide*<2->{
    初めから fixup する予定で commit したのなら

    commit message に更新内容を書く必要は無い

    どの commit に fixup する予定かを示せば十分
  }
  \vspace{2ex}

  \onslide*<3->{
    \$ git commit {\dhyphen}fixup=\textit{commit}

    とすると、
    \vspace{2ex}

    ``fixup! {\textless}\textit{commit} の message\textgreater''

    という commit message をつけて commit してくれる
  }
  \vspace{2ex}

  \onslide*<4->{
    同様の

    \$ git commit {\dhyphen}squash=\textit{commit}

    というコマンドもある
  }
  \vspace{2ex}

\end{frame}


\begin{frame}[t]{過去の履歴を書き換える}{}
  個人的には、commit の指定には

  圧倒的に HEAD が多い
  \vspace{4ex}

  git commit {\dhyphen}fixup=HEAD

  や

  git commit {\dhyphen}squash=HEAD
  \vspace{2ex}

  は長いので、alias を張ってます
\end{frame}


\begin{frame}[t]{過去の履歴を書き換える}{}
  \$ git config {\dhyphen}global alias.fixup {\bslash}

  ~~~~``commit {\dhyphen}fixup=HEAD''

  \$ git config {\dhyphen}global alias.squash {\bslash}

  ~~~~``commit {\dhyphen}squash=HEAD''
  \vspace{2ex}

  \onslide*<2->{
    こうすると、

    \$ git fixup (\$ git squash)

    と実行するだけで
    \vspace{2ex}

    git commit {\dhyphen}fixup=HEAD

    (git commit {\dhyphen}squash=HEAD)
    \vspace{2ex}

    と同じ意味になる
  }
\end{frame}


\begin{frame}[t]{過去の履歴を書き換える}{}

  さらに言うと、
  \vspace{2ex}

  \onslide*<2->{
    commit message に fixup や squash の印が

    フォーマットに従って記載してあれば、

    rebase -i する時に f や s の印をつける事を

    機械的に判断可能
  }
  \vspace{2ex}

  \onslide*<3->{じゃあ、機械にやってもらいましょう}

\end{frame}


\begin{frame}[t]{過去の履歴を書き換える}{}
  \$ git rebase -i \textit{commit} {\dhyphen}autosquash
  \vspace{4ex}


  先頭が fixup! や squash! で始まり

  続く commit message が commit 再編集の対象の場合
  \vspace{2ex}


  commit message に従って

  fixup や squash の印をつけ、

  commit の順番入れ替えまで実行済みの状態で

  rebase -i のエディタが開く
\end{frame}


\begin{frame}[t]{過去の履歴を書き換える}{}
  \$ git rebase -i \textit{commit} {\dhyphen}autosquash
  \vspace{2ex}

  も長い?
  \vspace{4ex}

  \onslide*<2>{
    \$ git config {\dhyphen}global rebase.autosquash true
    \vspace{2ex}

    とすると、

    rebase -i 時に毎回 {\dhyphen}autosquash をデフォルトでつけてくれる
  }
\end{frame}


\begin{frame}[t]{過去の履歴を書き換える}{}

  良く言われるように、git は branch 作成コストが低く

  master に影響を与える事なく、実験が出来る
  \vspace{2ex}

  \onslide*<2->{
    実験の結果、うまくいった commit を取り込み、

    不要な commit を外すのにも rebase -i は便利
    \vspace{2ex}

    \onslide<3->{フィードバックを行うまでが実験!}
  }
  \vspace{2ex}

  \onslide*<4->{
    個人的には、push しない事前提に

    デバッグ用のコード (print 文等) を仕込んだ

    commit も好き
  }
\end{frame}
