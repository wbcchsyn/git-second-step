% Copyright 2014 Yoshida Shin
% 
% This is part of "Git Second Step."
% 
% This program is free software: you can redistribute it and/or modify
%     it under the terms of the GNU General Public License as published by
%     the Free Software Foundation, either version 3 of the License, or
%     (at your option) any later version.
% 
%     This program is distributed in the hope that it will be useful,
%     but WITHOUT ANY WARRANTY; without even the implied warranty of
%     MERCHANTABILITY or FITNESS FOR A PARTICULAR PURPOSE.  See the
%     GNU General Public License for more details.
% 
%     You should have received a copy of the GNU General Public License
%     along with this program.  If not, see <http://www.gnu.org/licenses/>.

\begin{frame}[t]{first forward merge できない場合}{}

  \begin{columns}

    \begin{narrowcolumn}

      \begin{center}
        \begin{tikzpicture} [x=1em, y=1ex]

          \onslide<11-13,16->{
            \node (d11) [commit] at (4, 22) {A};
            \node (d9) [commit] at (4, 18) {};
          }

          \onslide*<2-5>{\node (m8) [commit] at (0, 16) {A};}

          \onslide*<9-10>{\node (d7) [commit] at (4, 14) {A B};}
          \onslide*<1-8,11->{\node (d7) [commit] at (4, 14) {B};}
          \node (d5) [commit] at (4, 10) {};
          \node (d3) [commit] at (4, 6) {};

          \onslide*<2-5,9-13,16->{\node (m6) [commit] at (0, 12) {};}
          \onslide*<1,6-8,14-15>{\node (m6) [commit] at (0, 12) {A};}
          \node (m4) [commit] at (0, 8) {};
          \node (m2) [commit] at (0, 4) {};
          \node (m0) [commit] at (0, 0) {};

          \onslide*<11-13,16->{
            \draw [line width=3pt, color = red] (d9) -- (d11);
            \draw [line width=3pt, color = green] (d7) -- (d9);
          }

          \onslide*<7-13,16->{
            \draw [line width=3pt, color = red] (m4) -- (m6);
            \draw [line width=3pt, color = green] (m2) -- (m4);
          }

          \onslide*<2-5>{
            \draw (d7) -- (m8);
            \draw (m6) -- (m8);
          }

          \draw (d5) -- (d7);
          \draw (d3) -- (d5);
          \draw (m2) -- (d3);

          \onslide*<-6,14-15>{
            \draw (m4) -- (m6);
            \draw (m2) -- (m4);
          }
          \draw (m0) -- (m2);

        \end{tikzpicture}

        {\footnotesize{(A \alert{$\leftarrow$ HEAD})}}
      \end{center}

    \end{narrowcolumn}

    \begin{widecolumn}

      \onslide*<1-5>{
        方法1

        気にせず no first forward merge

        \onslide<3->{
          \begin{itemize}
          \item 簡単に実行可能
            \onslide<4->{
            \item よーく考えると

              \onslide<5>{全然解決になっていない}
            }
          \end{itemize}
        }
      }

      \onslide*<6-13>{
        方法2

        git diff により A 上の

        commit のパッチを作成
        \vspace{2ex}

        \onslide*<8->{A の向き先を B へ変更}
        \vspace{2ex}

        \onslide<10->{
          作ったパッチを適用し、

          新しく commit
        }

        \onslide<12->{
          \begin{itemize}
          \item 履歴が 1直線になる
            \onslide<13->{\item パッチ適用の数だけ衝突の可能性あり}
            \onslide<14->{\item 手作業が辛い}
          \end{itemize}
        }
      }

      \onslide*<14-18>{
        方法3

        方法2 を自動化する

        \onslide<15->{\$ git rebase A}

        \onslide<17->{
          \begin{itemize}
          \item 履歴が 1直線になる
            \onslide<18->{\item commit の数だけ衝突する可能性あり}
          \end{itemize}
        }
      }

      \onslide*<19->{
        図から分かるように、rebase により

        commit が消える事は無い
        \vspace{2ex}

        \onslide*<20->{branch から外れるだけ}
        \vspace{2ex}

        \onslide*<21->{
          gc の前ならば、切り戻し可能

          (デフォルトの gc では、数日間は消されない)
        }
        \vspace{2ex}

        \onslide*<22->{
          rebase に限らず、git では一般的に

          commit がその場で消える事は無い

          多くの場合、切り戻し可能
        }
      }
    \end{widecolumn}
  \end{columns}

\end{frame}
