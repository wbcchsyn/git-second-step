% Copyright 2014 Yoshida Shin
% 
% This is part of "Git Second Step."
% 
% This program is free software: you can redistribute it and/or modify
%     it under the terms of the GNU General Public License as published by
%     the Free Software Foundation, either version 3 of the License, or
%     (at your option) any later version.
% 
%     This program is distributed in the hope that it will be useful,
%     but WITHOUT ANY WARRANTY; without even the implied warranty of
%     MERCHANTABILITY or FITNESS FOR A PARTICULAR PURPOSE.  See the
%     GNU General Public License for more details.
% 
%     You should have received a copy of the GNU General Public License
%     along with this program.  If not, see <http://www.gnu.org/licenses/>.

\begin{frame}{}{}
  merge と rebase の使用方法例
\end{frame}

\begin{frame}[t]{branch 輪廻から解脱}{}
  \begin{center}
    \begin{tikzpicture} [x=1em, y=1ex]
      \onslide<17->{\node (3) [commit] at (8,20) {master};}

      \onslide<9->{
        \onslide<13-14>{\node (E) [commit] at (16,16) {master devel};}
        \onslide*<15->{\node (E) [commit] at (16,16) {devel};}
        \onslide*<-12>{\node (E) [commit] at (16,16) {master};}
        \node (D) [commit] at (16,12) {};
      }

      \onslide<15-16>{
        \node (2) [commit, rectangle split, rectangle split parts=2] at (8,8){
          {master}
          \nodepart{second}{\tiny remotes/origin/master}
        };
      }
      \onslide*<7-14,17->{\node (2) [commit] at (8,8) {remotes/origin/master};}
      \onslide*<7->{\node (1) [commit] at (8,4) {};}

      \onslide*<3-4>{\node (D) [commit] at (0,12) {master};}

      \onslide*<3-8>{
        \node (C) [commit] at (0,8) {\onslide*<5->{master}};
        \node (B) [commit] at (0,4) {};
      }

      \onslide<-2>{\node (A) [commit] at (0,0) {master};}
      \onslide*<3->{\node (A) [commit] at (0,0) {};}

      \onslide*<17->{
        \draw (E) -- (3);
        \draw (2) -- (3);
      }

      \onslide*<9->{
        \draw (D) -- (E);
        \draw (2) -- (D);
      }

      \onslide*<7->{
        \draw (1) -- (2);
        \draw (A) -- (1);
      }

      \onslide*<3-4>{\draw (C) -- (D);}

      \onslide*<3-8>{
        \draw (B) -- (C);
        \draw (A) -- (B);
      }
    \end{tikzpicture}
  \end{center}

  \onslide*<1>{
    remote から clone したレポジトリで

    他の人と共同開発する場合
  }

  \onslide*<2-3>{
    master で開発開始
  }

  \onslide*<4-5>{
    master の履歴を整理
    \vspace{4ex}

    \$ git rebase -i ***
  }

  \onslide*<6-7>{
    remote branch をアップデート

    \$ git fetch {\dhyphen}all -p
  }

  \onslide*<8-9>{
    remote branch の更新を master に反映

    \$ git rebase remotes/origin/master
  }

  \onslide*<10>{
    commit tree を 1本にしたい場合は

    これで終了
  }

  \onslide*<11>{
    github で pull request 出す時は、

    この状態で出すと良い
  }

  \onslide*<12-13>{
    commit tree を 2本にする場合は

    現在の HEAD に適当な新しいブランチ作成

    \$ git branch devel
  }

  \onslide*<14-15>{
    HEAD を remote branch の最新状態まで

    巻き戻す

    \$ git reset remotes/origin/master
  }

  \onslide*<16-17>{
    devel を no first forward で merge

    \$ git merge devel {\dhyphen}no-ff
  }

\end{frame}


\begin{frame}[t]{branch 輪廻から解脱}{}

  ところで

  「no first forward の merge をする場合は

   rebase 必要ないのでは?」

  と思った人
  \vspace{4ex}

  \onslide<2->{\alert{鋭い}}
  \vspace{4ex}

  \onslide<3->{
    と見せて

    \alert{あまい}
  }

\end{frame}


\begin{frame}<1-13>[t]{branch 輪廻から解脱}{}

  自分が作業している間に remote branch が更新されたら

  \begin{columns}

    \begin{narrowcolumn}
      \begin{center}
        \begin{tikzpicture} [x=1em, y=1ex]

          \onslide<99>{\node (dummy) [commit] at (0, 16) {};}

          \onslide<3->{\node (C) [commit] at (0, 12) {};}

          \node (2) [commit] at (3,8) {};
          \node (1) [commit] at (3,4) {};
          \node (B) [commit] at (0,4) {};
          \node (A) [commit] at (0,0) {};

          \onslide<3->{
            \draw (2) -- (C);
            \draw (B) -- (C);
          }

          \draw (1) -- (2);
          \draw (A) -- (1);
          \draw (A) -- (B);
        \end{tikzpicture}

        merge のみ
      \end{center}
    \end{narrowcolumn}

    \begin{narrowcolumn}
      \begin{center}
        \begin{tikzpicture} [x=1em, y=1ex]

          \onslide<99>{\node (dummy) [commit] at (0, 16) {};}

          \onslide<7->{
            \node (2) [commit] at (3,12) {};
            \node (1) [commit] at (3,8) {};
          }

          \onslide*<-6>{
            \node (2) [commit] at (3,8) {};
            \node (1) [commit] at (3,4) {};
          }
          \node (B) [commit] at (0,4) {};
          \node (A) [commit] at (0,0) {};

          \onslide<7->{\draw (B) -- (1);}
          \draw (1) -- (2);
          \onslide<-6>{\draw (A) -- (1);}
          \draw (A) -- (B);
        \end{tikzpicture}

        rebase のみ
      \end{center}
    \end{narrowcolumn}

    \begin{narrowcolumn}
      \begin{center}
        \begin{tikzpicture} [x=1em, y=1ex]
          \onslide<10->{
            \node (C) [commit] at (0,16) {};
            \node (2) [commit] at (3,12) {};
            \node (1) [commit] at (3,8) {};
          }

          \onslide*<-9>{
            \node (2) [commit] at (3,8) {};
            \node (1) [commit] at (3,4) {};
          }
          \node (B) [commit] at (0,4) {};
          \node (A) [commit] at (0,0) {};

          \onslide<10->{
            \draw (2) -- (C);
            \draw (B) -- (C);
            \draw (B) -- (1);
          }
          \draw (1) -- (2);
          \onslide<-9>{\draw (A) -- (1);}
          \draw (A) -- (B);
        \end{tikzpicture}

        rebase + merge
      \end{center}
    \end{narrowcolumn}

  \end{columns}
  \vspace{2ex}

  \onslide*<2>{rebase せずに merge で取り込む}

  \onslide*<4-5>{
    \alert{今回は} branch が 2又に分かれる

    \onslide*<5>{
      より正確には、開発ラインの数 (= 開発者の数?)だけ

      branch が分かれ、merge が入り乱れる
    }
  }

  \onslide*<6>{rebase で取り込む}

  \onslide*<8>{開発ラインの数によらず、branch は 1本}

  \onslide*<9>{rebase してから no first forward merge で取り込む}

  \onslide*<11>{
    開発ラインの数によらず、

    更新の有る branch は常に 1本
  }


  \onslide*<12>{
    その他、更新の無い branch が最大 1本

    branch が合計 3本以上になる事は無い
  }

  \onslide*<13>{
    結論

    branch のアミダクジ状態を避ける為には

    rebase は必要
  }
\end{frame}
