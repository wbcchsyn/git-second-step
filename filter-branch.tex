% Copyright 2014 Yoshida Shin
% 
% This is part of "Git Second Step."
% 
% This program is free software: you can redistribute it and/or modify
%     it under the terms of the GNU General Public License as published by
%     the Free Software Foundation, either version 3 of the License, or
%     (at your option) any later version.
% 
%     This program is distributed in the hope that it will be useful,
%     but WITHOUT ANY WARRANTY; without even the implied warranty of
%     MERCHANTABILITY or FITNESS FOR A PARTICULAR PURPOSE.  See the
%     GNU General Public License for more details.
% 
%     You should have received a copy of the GNU General Public License
%     along with this program.  If not, see <http://www.gnu.org/licenses/>.

\begin{frame}[t]{git の入れ子管理}{repository の分離}
  git submodule が他 repository を特定ディレクトリに

  対応させるコマンドなら、
  \vspace{2ex}

  特定ディレクトリを独立した repository に

  変換するコマンドは

  git filter-branch
\end{frame}


\begin{frame}[t]{git の入れ子管理}{repository の分離}

  workspace のトップディレクトリで
  \vspace{2ex}

  \$ git filter-branch {\dhyphen}subdirectory-filter {\bslash}
  ~~~~\textit{path\_to\_the\_directory} {\dhyphen} {\dhyphen}all
  \vspace{4ex}

  \onslide*<2->{
    reflog も新しくなるので注意

    (reflog で過去の commit を探して切り戻す事は不可能)
  }
  \vspace{2ex}

  \onslide*<3->{tag も残るが、中身は変更される}
  \vspace{2ex}

  \onslide*<3->{
    ただし、filter-branch 実行前の branch だけなら

    .git/refs/original/refs/head 以下のテキストファイルに

    ハッシュ値が記載

    そこから復旧可能
  }
\end{frame}

\begin{frame}[t]{git の入れ子管理}{}
  ここでは紹介しきれていないが

  submodule も filter-branch も癖が有るコマンドなので

  要注意
  \vspace{4ex}

  実行する前に man を良く確認し、

  バックアップを取ってから

  実行する事を推奨
  \vspace{4ex}

  でも、repository を細かく管理できるのは

  それを補ってあまり有る利便性(だと思う)
\end{frame}
