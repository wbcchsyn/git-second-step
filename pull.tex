% Copyright 2014 Yoshida Shin
% 
% This is part of "Git Second Step."
% 
% This program is free software: you can redistribute it and/or modify
%     it under the terms of the GNU General Public License as published by
%     the Free Software Foundation, either version 3 of the License, or
%     (at your option) any later version.
% 
%     This program is distributed in the hope that it will be useful,
%     but WITHOUT ANY WARRANTY; without even the implied warranty of
%     MERCHANTABILITY or FITNESS FOR A PARTICULAR PURPOSE.  See the
%     GNU General Public License for more details.
% 
%     You should have received a copy of the GNU General Public License
%     along with this program.  If not, see <http://www.gnu.org/licenses/>.

\begin{frame}{}{}
  今更 git pull
\end{frame}


\begin{frame}[t]{今更 git pull}{}
  ネットでは

  「git を良く知らない間は pull はやめた方が良い」

  と良く言われる
  \vspace{4ex}

  \onslide*<2>{
    その割には、

    良く知っている人はどうやって使うのか

    あまり記載が無いので、私なりの回答をご紹介
  }

  \onslide*<3->{なぜやめた方が良いのか?}
  \vspace{4ex}

  \onslide*<4->{
    git pull はリモートブランチの更新を取り込む

    shell script
  }
  \vspace{2ex}

  \onslide*<5->{デフォルトの挙動は、fetch + merge}
  \vspace{2ex}

  \onslide*<6->{merge のアンチパターン直撃}
\end{frame}


\begin{frame}[t]{今更 git pull}{}
  git pull との上手な接し方
  \onslide*<2->{
    \begin{enumerate}
    \item 使わない
      \onslide*<3->{\item 挙動を fetch + rebase に変更}
      \onslide*<4->{\item no first forward merge を禁止}
    \end{enumerate}
  }
\end{frame}


\begin{frame}[t]{今更 git pull}{}
  git pull 使わない
  \vspace{2ex}

  \begin{itemize}
  \item fetch + rebase で十分
  \item remote の更新確認は頻繁に行う

    しかし、更新取り込みは別のタイミングで行いたい

    更新確認の度に衝突の危機になったり、

    自分のコードが変更されるるのは非生産的

    思考を中断しない為に、結局 fetch を使う
  \end{itemize}
  \vspace{2ex}

  \onslide*<2>{私は、普段この方法}
\end{frame}


\begin{frame}[t]{今更 git pull}{}
  挙動を fetch + rebase にする
  \vspace{2ex}

  \onslide*<1->{
    \begin{itemize}
    \item 手軽に remote の更新を local に反映できる
    \item 衝突が発生しやすい
    \end{itemize}
  }
  \vspace{2ex}

  \onslide*<2->{
    設定方法

    \$ git config {\dhyphen}global pull.rebase true
  }
  \vspace{2ex}

  \onslide*<3>{簡単な開発の場合は、私はこの方法を使います}
\end{frame}


\begin{frame}[t]{今更 git pull}{}
  no first forward merge を禁止
  \vspace{2ex}

  \onslide*<1->{
    \begin{itemize}
    \item local での開発には開発用 branch で行う必要あり
    \item pull に限らず、no first foward を禁止すると

      merge に関わる事故や問題の多くを回避可能

      うっかりミス防止のため、システム的に禁止
    \item no first forward merge は、有るから使う程度

      個人的には、無くても大きな問題は無い気がする
    \end{itemize}
  }
  \vspace{2ex}

  \onslide*<2->{
    設定方法

    \$ git config {\dhyphen}global merge.ff only
  }
  \vspace{2ex}

  \onslide*<3>{昔、この方法を使っていました}
\end{frame}


\begin{frame}[t]{今更 git pull}{}
  結論
  \vspace{4ex}

  好きな方法で良い
  \vspace{2ex}

  ただし、remote repository は汚すな
\end{frame}
