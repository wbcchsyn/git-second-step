% Copyright 2014 Yoshida Shin
% 
% This is part of "Git Second Step."
% 
% This program is free software: you can redistribute it and/or modify
%     it under the terms of the GNU General Public License as published by
%     the Free Software Foundation, either version 3 of the License, or
%     (at your option) any later version.
% 
%     This program is distributed in the hope that it will be useful,
%     but WITHOUT ANY WARRANTY; without even the implied warranty of
%     MERCHANTABILITY or FITNESS FOR A PARTICULAR PURPOSE.  See the
%     GNU General Public License for more details.
% 
%     You should have received a copy of the GNU General Public License
%     along with this program.  If not, see <http://www.gnu.org/licenses/>.

\begin{frame}{}{}
  特定の commit だけ取り込む
\end{frame}

\begin{frame}[t]{特定の commit だけ取り込む}{}

  \begin{columns}

    \begin{narrowcolumn}

      \begin{center}
        \begin{tikzpicture}[x=1em, y=1ex]

          \onslide<5->{\node (F') [commit] at (0, 16) {F'};}

          \node (G) [commit] at (4, 14) {G};
          \node (F) [commit] at (4, 10) {F};
          \node (E) [commit] at (4, 6) {E};

          \node (D) [commit] at (0, 12) {D};
          \node (C) [commit] at (0, 8) {C};
          \node (B) [commit] at (0, 4) {B};
          \node (A) [commit] at (0, 0) {A};

          \draw (F) -- (G);
          \draw (E) -- (F);
          \draw (B) -- (E);

          \onslide*<5->{\draw (D) -- (F');}
          \draw (C) -- (D);
          \draw (B) -- (C);
          \draw (A) -- (B);
        \end{tikzpicture}

        \onslide*<-4>{\footnotesize{(D \alert{$\leftarrow$ HEAD})}}

        \onslide*<-4>{\footnotesize{(F' \alert{$\leftarrow$ HEAD})}}
      \end{center}
    \end{narrowcolumn}

    \begin{widecolumn}
      \onslide*<1-3>{D に F のパッチだけ適用
        \vspace{4ex}

        \onslide*<2->{
          rebase や merge を使うと E, F, G が

          全て取り込まれるので駄目
          \vspace{2ex}

          \onslide<3>{
            代わりに cherry-pick を使用
          }
        }
      }

      \onslide*<4-5>{
        \$ git cherry-pick -x F
        \vspace{2ex}

        (F の指定方法はハッシュ値等)
        \vspace{2ex}

        (branch 名等を使用すると

        ~別の挙動を示すので注意)
      }


      \onslide*<6-7>{
        F' の tree に

        必要な G のパッチを

        全て当てたか確認
        \vspace{2ex}

        \onslide<7>{
          git log の比較だけでは大変
        }
      }
      \onslide*<8-10>{
        \$ git cherry -v F' G

        (F' や G の指定方法は

        ~branch, HEAD, ハッシュ値等)
        \vspace{2ex}

        \code{
          {\scriptsize
            + a0206cdf1ebe2c866c68e2dfaed68087dabc9fbc E

            - a5abacf9b353d20210d88679e0b4f10ca8ee21ea F

            + 3b6b43ffb099970e9f328c5c8163ee174e68eaef G
          }
        }
        \vspace{2ex}

        \onslide*<9>{
          + は F' の先祖に無いが

          G の先祖に有る commit
        }
        \onslide*<10>{
          - は F' の先祖に無いが

          パッチが取り込み済みの commit

          (F と F' は別の commit)
         }
       }
     \end{widecolumn}

   \end{columns}

 \end{frame}


 \begin{frame}[t]{特定の commit だけ取り込む}{}

   git cherry-pick はセキュリティの緊急対応等を

   全 branch に適用するのに便利

   \onslide*<2->{
     \begin{enumerate}
     \item バージョンごとに branch を切って管理している

       普段は最新バージョンの branch しか更新しないが、

       セキュリティ対応のパッチだけは全 branch に当てたい
       \onslide*<3->{
       \item 通常、開発は devel で、リリースは master で実施

         今回は緊急のため master で直接 commit

         このパッチを後から devel に取り込む必要あり
       }
     \end{enumerate}
   }

 \end{frame}
