% Copyright 2014 Yoshida Shin
% 
% This is part of "Git Second Step."
% 
% This program is free software: you can redistribute it and/or modify
%     it under the terms of the GNU General Public License as published by
%     the Free Software Foundation, either version 3 of the License, or
%     (at your option) any later version.
% 
%     This program is distributed in the hope that it will be useful,
%     but WITHOUT ANY WARRANTY; without even the implied warranty of
%     MERCHANTABILITY or FITNESS FOR A PARTICULAR PURPOSE.  See the
%     GNU General Public License for more details.
% 
%     You should have received a copy of the GNU General Public License
%     along with this program.  If not, see <http://www.gnu.org/licenses/>.

\begin{frame}{}{}
  未 commit の作業を一時中断する
\end{frame}

\begin{frame}[t]{未 commit の作業を一時中断する}{}

  \onslide*<1-2>{
    rebase や cherry-pick は便利だが、

    workspace と cached が clean な時しか実行不可

    (git status で差分が出てきて駄目)

    checkout や merge も clean な状況で行った方が安全

    \vspace{4ex}

    \onslide<2->{
      未 commit のファイルがある状態で

      rebase や cherry-pick を行うには

      どうしたら良いか?
    }
  }

  \onslide*<3-8>{
    \begin{enumerate}
    \item 未 commit のファイルを退避
      \onslide<4->{\item git reset {\dhyphen}hard 実行}
      \onslide<5->{\item rebase や cherry-pick 実行}
      \onslide<6->{\item 退避させたファイルを戻す}
    \end{enumerate}
    \vspace{4ex}

    \onslide<7->{手作業が面倒なので、自動化したい}

    \onslide<8->{そこで stash}
  }

  \onslide*<9-10>{
    \$ git stash
    \vspace{2ex}

    workspace と cached の変更をスタックし

    clean にする
    \vspace{4ex}

    \onslide<10>{
      \$ git stash pop
      \vspace{2ex}

      スタックから workspace と cached の変更を取り出し

      適用する
    }
  }

  \onslide*<11-15>{
    未 commit のファイルがある状態で rebase や cherry-pick を行う

    (stash 使用バージョン)
    \vspace{2ex}

    \onslide<12->{
      \begin{enumerate}
      \item \$ git stash
        \onslide<13->{\item rebase や cherry-pick 実行}
        \onslide<14->{\item \$ git stash pop}
      \end{enumerate}
      \vspace{2ex}

      \onslide<15->{merge や checckout で使っても便利}
    }
  }

  \onslide*<16>{
    ちなみに、
    \vspace{2ex}

    \$ git config {\dhyphen}global rebase.autostash true
    \vspace{2ex}

    とすると、
    \vspace{4ex}

    git rebase と実行するだけで
    \vspace{2ex}

    stash $\rightarrow$ rebase $\rightarrow$ stash pop
    \vspace{2ex}

    を実行してくれるようになる
  }

  \onslide*<17->{
    補足
    \vspace{2ex}

    stash は複数の変更をスタック、保存可能
    \vspace{2ex}

    stash した branch とは別の branch で pop する事も可能
    \vspace{2ex}

    でも、やり過ぎるとかえって頭が混乱するので程々に

    (stash pop で衝突とかすると、泣きたくなります)
    \vspace{2ex}

    複雑になる場合は、一時 branch 作った方が楽かも
  }
\end{frame}
