% Copyright 2014 Yoshida Shin
% 
% This is part of "Git Second Step."
% 
% This program is free software: you can redistribute it and/or modify
%     it under the terms of the GNU General Public License as published by
%     the Free Software Foundation, either version 3 of the License, or
%     (at your option) any later version.
% 
%     This program is distributed in the hope that it will be useful,
%     but WITHOUT ANY WARRANTY; without even the implied warranty of
%     MERCHANTABILITY or FITNESS FOR A PARTICULAR PURPOSE.  See the
%     GNU General Public License for more details.
% 
%     You should have received a copy of the GNU General Public License
%     along with this program.  If not, see <http://www.gnu.org/licenses/>.

\begin{frame}{}{}
  subversion から git へ
\end{frame}


\begin{frame}[t]{subversion から git へ}{}

  個人的に subversion から git へ移行して変わった事
  \onslide*<2->{
    \begin{enumerate}
    \item commit が細かくなる
    \item push しない前提の commit が可能
    \end{enumerate}
  }
\end{frame}


\begin{frame}[t]{subversion から git へ}{commit が細かくなる}

  \begin{enumerate}
  \item push 前に直せば良いので、気軽に commit
    \onslide*<2->{
    \item commit が小さく、多くなる
      \onslide*<3->{
        \begin{itemize}
        \item 開発中に細かい切り戻しが可能
        \item 必要なパッチだけ、ピンポイントで選択可能
        \item 問題のある commit を特定するだけで

          調査範囲を狭められる
        \end{itemize}
      }
    }
    \onslide*<4->{\item commit をもっと小さくしたくなる}
    \onslide*<5->{
      \begin{itemize}
      \item test を書くようになる

        (commit 回数が多いと test 実行回数も増えるので

        ~test の効果が高い)
      \item 後から変更しやすいコードを意識するようになる

        小さいパッチで commit するには、

        最初から変更に耐えるコードである必要がある

        (もし、そうでないならリファクタリングを挟む)
      \end{itemize}
    }
  \end{enumerate}

\end{frame}
