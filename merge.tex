% Copyright 2014 Yoshida Shin
% 
% This is part of "Git Second Step."
% 
% This program is free software: you can redistribute it and/or modify
%     it under the terms of the GNU General Public License as published by
%     the Free Software Foundation, either version 3 of the License, or
%     (at your option) any later version.
% 
%     This program is distributed in the hope that it will be useful,
%     but WITHOUT ANY WARRANTY; without even the implied warranty of
%     MERCHANTABILITY or FITNESS FOR A PARTICULAR PURPOSE.  See the
%     GNU General Public License for more details.
% 
%     You should have received a copy of the GNU General Public License
%     along with this program.  If not, see <http://www.gnu.org/licenses/>.

\begin{frame}{}{}
  2種類の merge と rebase
\end{frame}

\begin{frame}[t]{2種類の merge}{}

  git の merge には

  ``first forward'' と ``no first forward''

  の2種類が存在
  \vspace{4ex}

  デフォルトではその時の branch の状況によって

  自動で使い分けられる
  \vspace{2ex}

  オプションを指定する事である程度制御可能
\end{frame}


\begin{frame}[t]{2種類の merge}{HEAD が merge 対象の先祖の時}

  \begin{columns}

    \begin{narrowcolumn}

      \begin{center}
        \begin{tikzpicture} [x=1em, y=1ex]

          \onslide<3->{\node (d4) [commit] at (4, 16) {A B};}
          \onslide*<1-2>{\node (d4) [commit] at (4, 16) {B};}
          \node (d3) [commit] at (4, 12) {};
          \onslide*<-2>{\node (m2) [commit] at (0, 8)  {A};}
          \onslide*<3->{\node (m2) [commit] at (0, 8)  {};}
          \node (m1) [commit] at (0, 4) {};
          \node (m0) [commit] at (0, 0) {};

          \draw (d3) -- (d4);
          \draw (m2) -- (d3);
          \draw (m1) -- (m2);
          \draw (m0) -- (m1);

        \end{tikzpicture}

        {\footnotesize{(A \alert{$\leftarrow$ HEAD})}}
      \end{center}

    \end{narrowcolumn}

    \begin{widecolumn}

      \onslide*<1>{
        A は B の先祖の commit
      }

      \onslide*<2-4>{
        \$ git merge B
        \vspace{2ex}

        \onslide*<4>{branch の向き先変更だけ}
      }

      \onslide*<5>{
        新しい commit が作られないので

        commit message の入力も無し
        \vspace{2ex}

        このように、

        新しい commit を作らない merge が

        first forward merge
      }

      \onslide*<6>{
        ちなみに、push では通常

        first forward merge を使用
      }
    \end{widecolumn}

  \end{columns}

\end{frame}


\begin{frame}[t]{2種類の merge}{HEAD が merge 対象が先祖では無い}

  \begin{columns}

    \begin{narrowcolumn}

      \begin{center}
        \begin{tikzpicture} [x=1em, y=1ex]

          \onslide<9->{\node (m7) [commit] at (0, 28) {A};}
          \onslide*<4-8>{\node (m7) [commit] at (0, 28) {}};

          \node (d6) [commit] at (4, 24) {B};
          \node (d5) [commit] at (4, 20) {};
          \node (d2) [commit] at (4, 8) {};

          \node (m4) [commit] at (0, 16) {\onslide*<-8>{A}};
          \node (m3) [commit] at (0, 12) {};
          \node (m1) [commit] at (0, 4) {};
          \node (m0) [commit] at (0, 0) {};

          \draw (d5) -- (d6);
          \draw (d2) -- (d5);
          \draw (m1) -- (d2);

          \draw (m3) -- (m4);
          \draw (m1) -- (m3);
          \draw (m0) -- (m1);

          \onslide*<6->{
            \draw [line width=2pt, color=green, ->] (m1) -- (m3) -- (m4);
            \draw [line width=2pt, color=green, ->] (d6) -- (m7);
          }
          \onslide*<4->{
            \draw [line width=2pt, color=red, ->] (m4) -- (m7);
          }
          \onslide*<3->{
            \draw [line width=2pt, color=red, ->] (m1) -- (d2) -- (d5) -- (d6);
          }
        \end{tikzpicture}

        {\footnotesize{(A \alert{$\leftarrow$ HEAD})}}
      \end{center}

    \end{narrowcolumn}

    \begin{widecolumn}

      \onslide*<1>{
        A と B は branch の分岐後

        それぞれ開発
      }

      \onslide*<2-4>{
        \$ git merge B
        \vspace{2ex}

        \onslide*<3-4>{
          B の更新分のパッチを

          A に適用し、

          新しい commit object を作成}
      }

      \onslide*<5-6>{
        衝突が無ければ、A のパッチを

        B に適用しても同じ物になるはず
      }

      \onslide*<7>{
        新しい commit を作るので

        commit message を入力
      }

      \onslide*<8-9>{
        HEAD を新しい commit へ移動
      }

      \onslide*<10>{
        このように、

        新しい commit が作られる merge が

        no first forward merge
      }
    \end{widecolumn}

  \end{columns}

\end{frame}


\begin{frame}[t]{2種類の merge}{commit tree が 2又の時}

  \begin{columns}

    \begin{narrowcolumn}

      \begin{center}
        \begin{tikzpicture} [x=1em, y=1ex]

          \node (m7) [commit] at (0, 28) {A};

          \node (d6) [commit] at (4, 24) {B};
          \node (d5) [commit] at (4, 20) {};
          \node (d2) [commit] at (4, 8) {};

          \node (m4) [commit] at (0, 16) {};
          \node (m3) [commit] at (0, 12) {};
          \onslide*<4-6>{\node (m3) [commit] at (0, 12) {\alert{?}};}
          \node (m1) [commit] at (0, 4) {};
          \node (m0) [commit] at (0, 0) {};


          \onslide*<1,8->{
            \draw (d6) -- (m7);
            \draw (d5) -- (d6);
            \draw (d2) -- (d5);
            \draw (m1) -- (d2);

            \draw (m4) -- (m7);
            \draw (m3) -- (m4);
            \draw (m1) -- (m3);
            \draw (m0) -- (m1);
          }

          \onslide*<2-7>{
            \draw (d6) -- (m7);
            \draw (d5) -- (d6);
            \draw (m4) -- (d5);
            \draw (m3) -- (m4);
            \draw (d2) -- (m3);
            \draw (m1) -- (d2);
            \draw (m0) -- (m1);
          }
        \end{tikzpicture}
      \end{center}

    \end{narrowcolumn}

    \begin{widecolumn}
      \onslide*<1-2>{
        ちなみに、オプション無しで

        \$ git log A

        を実行すると、
      }

      \onslide*<3>{こう勘違いするかも}

      \onslide*<4-5>{
        例えばここ、
        \vspace{2ex}

        どうなってるの?
        \vspace{4ex}

        \onslide*<5>{
          B の 1個目の commit と

          A の 1個目の commit が

          merge されて......
          \vspace{4ex}

          あれ、衝突は?
        }
      }

      \onslide*<6>{どうもなっていません!}

      \onslide*<7->{
        あくまでも、正しいのはこっち
        \vspace{2ex}

        \onslide*<9->{
          \$ git log {\dhyphen}graph A
          \vspace{2ex}

          で確認可能
        }
        \vspace{2ex}

        \onslide*<10->{
          merge による変化は、

          commit 1個追加のみ
        }
        \vspace{2ex}

        \onslide*<11>{
          既存の commit は

          変化無し
        }
      }
    \end{widecolumn}
  \end{columns}
\end{frame}


\begin{frame}[t]{2種類の merge}{HEAD が merge 対象の先祖の時}

  \begin{columns}

    \begin{narrowcolumn}

      \begin{center}
        \begin{tikzpicture} [x=1em, y=1ex]

          \onslide<5->{\node (m5) [commit] at (0, 20) {A};}
          \onslide<3>{\node (d4) [commit] at (4, 16) {A B};}
          \onslide*<1-2,4->{\node (d4) [commit] at (4, 16) {B};}
          \node (d3) [commit] at (4, 12) {};
          \node (m2) [commit] at (0, 8)  {\onslide*<-2,4>{A}};
          \node (m1) [commit] at (0, 4) {};
          \node (m0) [commit] at (0, 0) {};


          \onslide*<5->{
            \draw (d4) -- (m5);
            \draw (m2) -- (m5);
          }
          \draw (d3) -- (d4);
          \draw (m2) -- (d3);
          \draw (m1) -- (m2);
          \draw (m0) -- (m1);

        \end{tikzpicture}

        {\footnotesize{(A \alert{$\leftarrow$ HEAD})}}
      \end{center}

    \end{narrowcolumn}

    \begin{widecolumn}
      \onslide*<1>{
        A が B の先祖の場合
        \vspace{4ex}

        merge コマンドに

        オプションを付与して

        merge 方法を選択可能
      }

      \onslide*<2-3>{
        {\dhyphen}ff-only オプションで

        first forward を強制
        \vspace{2ex}

        \$ git merge B {\dhyphen}ff-only
        \vspace{2ex}

        first forward できない場合に

        エラーになる
      }

      \onslide*<4-5>{
        {\dhyphen}no-ff オプションで

        no first foward を強制
        \vspace{2ex}

        \$ git merge {\dhyphen}no-ff B
        \vspace{2ex}

        HEAD が merge 対象の先祖でも

        無理矢理 no first forward で

        merge する
      }

      \onslide*<6>{
        デフォルトオプションは {\dhyphen}ff
        \vspace{2ex}

        可能であれば first forward,

        無理ならば no first forward
        \vspace{2ex}

        という意味
      }

    \end{widecolumn}

  \end{columns}

\end{frame}


\begin{frame}[t]{merge の種類とオプションまとめ}{}

  \begin{table}[htb]

    \begin{tabular}{c|ccc}
      \shortstack{\footnotesize HEAD が \\ \footnotesize 対象の} & {\small {\dhyphen}ff-only} & {\small {\dhyphen}ff (デフォルト)} & {\small {\dhyphen}no-ff} \\ \hline
      先祖 & {\small first forward} & {\small first forward} & {\small no first forward} \\
      & & & \\
      \shortstack{\footnotesize 先祖 \\ \footnotesize ではない} & エラー & {\small no first forward} & {\small no first forward}
    \end{tabular}
  \end{table}

\end{frame}


\begin{frame}[t]{2種類の merge の特徴}{first forward merge}
  first forward merge
  \vspace{2ex}

  \begin{itemize}
  \item 衝突等 merge に伴う事故が最も少ない
    \vspace{2ex}

  \item {\dhyphen}ff-only オプションで安全性の担保も可能
    \vspace{2ex}

  \item 過去の履歴が一直線なので簡単
  \end{itemize}
  \vspace{2ex}
\end{frame}


\begin{frame}[t]{2種類の merge の特徴}{no first forward merge}
  no first forward merge
  \vspace{2ex}

  \begin{itemize}
  \item branch だけで色々と主張できる
    \begin{itemize}
    \item 別 branch の複数の commit は関連する
    \item 別 branch の commit を 1個だけ適用すると

      中途半端な状況になる
    \end{itemize}
    \vspace{2ex}

  \item 過去の経緯が複雑
    \vspace{2ex}

  \item 親が複数有るので、パッチや履歴対象の操作が難しくなる\footnote{この欠点については今回の範囲を超えるので省略}
  \end{itemize}
\end{frame}


\begin{frame}[t]{merge の問題点}{}
  HEAD が merge 対象の先祖の場合は大きな問題は無い

  first forward と no first forward の好きな方を選べば良い
  \vspace{2ex}

  問題はそれ以外の場合

  \onslide<2->{特に、remote repository から更新を取り入れる時}
  \vspace{2ex}

  \onslide<3->{
    \begin{itemize}
    \item 「更新と取り込むため」だけの

      無意味な branch, commit が出来る

      (履歴を追う際のノイズになる)
      \onslide<4->{\item 複数 branch 間の merge は履歴が複雑}
    \end{itemize}
  }

  \onslide<5->{履歴を追えなければ SCM の意味が無い}
\end{frame}
